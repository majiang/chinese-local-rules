\documentclass{ltjsarticle}
\usepackage[ipaex]{luatexja-preset}
\usepackage{amsmath}

\title{太原麻将}
\begin{document}\maketitle
太原 (tai4yuan2) 麻将の「立牌」規則はこのルールを独特なものにしている。
配牌のうち最初の4枚は手牌の前に立て「立牌」と呼ぶ。
これらの牌はポン・カンに用いることができる (チーはまったくできない)。
これらの牌のうち任意の3枚を手牌の組み合わせに含める必要がある。
聴牌する際に、そのときに限り、これらの牌のうち1枚を裏向きに捨て、他家はこの牌にアクションできない。
立牌を捨てることによって聴牌を宣言しないとアガれない。
同巡の2鳴きは許されていない。
ポン・カンをする際には聴牌を宣言するための立牌を手に残さなければならない。
カンをするときは嶺上牌を補充する。
アガリは1局に1人まで (頭ハネ)。
アガリの見逃しは一切なし。
カンがなければ7トン、カン1つなら7トン、カン2つなら8トン、カン3つなら8トン、以下同様に残して、アガリが出ずに牌山を使い切ったら流局し連荘する。

\section{牌}
136枚を使用する。サイコロ2個。
\subsection{開局}
親の振ったサイコロの目で取牌対象の山を、2回目との和で取牌位置\footnote{从右至左とある}を決める。
\section{座位の決定}
子のアガリは輪荘、親のアガリは連荘。
流局は連荘だが、誰かがカンをしていたら輪荘。
\section{築牌}
17トン
\section{和牌型と役}
4面子1将頭の基本和牌型のみで特殊和牌型なし。
アガリに対する底点は1点で、以下の役があると加算される。
\begin{description}
    \item[缺一門] 1点
    \item[坎張] 1点
    \item[辺張] 1点
    \item[明槓] 1点
    \item[暗槓] 2点
    \item[親] 1点
    \item[清龍] 20点
    \item[摸和] 槓以外の点数を2倍
    \item[点和] 1点を加え、聴牌していなければ全員分を支払う。
\end{description}
\section{計算例}
\subsection{例}
親の自分が缺一門をツモアガったとき、得点は $(1+1)\times2 = 4$オールである。
\subsection{例}
親でない自分が聴牌者から清龍を出アガったとき、得点は $(1+1+20)+1 = 23$ である。
\end{document}

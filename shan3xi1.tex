\documentclass{ltjsarticle}
\usepackage[ipaex]{luatexja-preset}
\usepackage{amsmath}

\title{陝西麻将}
\begin{document}\maketitle
陝西 (shan3xi1) 麻将は基本和牌型はもちろん特殊和牌型でもアガることができる。
親の第一打の前に、全プレイヤーは炮子 (pao4zi3) を1個から4個、出すことができる。
炮子の個数に応じてこの局の得失点が増加するので、より刺激的である。
\section{牌}
136枚を使用する。サイコロ2個。
\section{座位の決定と開局}
まず親を決めてから全員の座位を決める。
親の振ったサイコロの目で取牌対象の山を、2回目との和で取牌位置\footnote{从左至右とあるが、内側から見て、か?}を決める。
\section{炮子・アガリ・流局}
親の第一打の前に、全プレイヤーは炮子\footnote{具体的な形状などは示されていないが、チップで代用してよいだろう}を1個から4個、出す\footnote{出す順については不明}ことができる。\footnote{出さなくてもよい}
各局のアガリは1人までである (頭ハネ)。
親がアガったら、または流局したら、その親を継続する。
さもなければその下家が親になる。
カンがなければ13トン、カン1つなら7トン、カン2つなら8トン、カン3つ以上なら13トン\footnote{13トンは13枚の誤りか? それにしても増えてから減るのは不自然である}を残して、アガリが出ずに牌山を使い切ったら流局する。
流局したら得失点はない。
\section{和牌型と役}
4面子1将頭の基本和牌型のほかに3種\footnote{4種の誤りだろう}の特殊和牌型があり、どの和牌型も底点は3点\footnote{全部この3点で割っても同値である。何か混乱があるのかもしれない}に固定されている。
\begin{enumerate}
    \item 十三幺
    \item 七対
    \item 全不靠
    \item 組合龍
\end{enumerate}
\section{得点計算}
\subsection{カンの得点}
\begin{description}
    \item[加槓] 全員から底点の1倍、すなわち3点をもらう。合計9点。
    \item[暗槓] 全員から底点の2倍、すなわち6点をもらう。合計18点。
    \item[大明槓] 打牌者から底点の2倍、すなわち6点をもらう。
\end{description}
\subsection{アガリの得点}
\begin{description}
    \item[摸和] 全員から底点の2倍、6点ずつもらう。
    \item[点和] 全員から底点の1倍、3点ずつもらう。
    \item[親] 親は2倍。
\end{description}
\subsection{炮子の点数}
アガリ者は、自分の炮子の個数の3倍に、他家の炮子の個数の和を加え、底点を乗じた点数をもらう。ただし、ツモアガリなら倍になる。
非アガリ者は、自分の炮子の個数に、アガリ者の炮子の個数を加え、底点を乗じた点数を支払う。\footnote{書かれていないが、摸和なら2倍、でないと計算が合わない}
\section{計算例}
\subsection{例}
親の自分が2個、他家がそれぞれ2個、3個、4個の炮子を出したとする。
あなたが対面からアガったとき、得点は $3\times2+(3\times2+(2+3+4))\times3 = 51$点である。\footnote{前節と矛盾している。前節の記述通りなら最初の$3\times2$も全員からもらうはずである}
\subsection{例}
親でない自分が4個、他家がそれぞれ2個 (親)、0個、1個の炮子を出したとする。
あなたが七対をツモアガったとき、得点は合計$204$点である。\footnote{やはり前章と整合性がとれない。}
\begin{description}
    \item[炮子2個の親] $3\times2\times2+(2+4)\times2\times6 = 84$点
    \item[炮子1個の他家] $3\times2+(1+4)\times2\times6 = 66$点
    \item[炮子0個の他家] $3\times2+(0+4)\times2\times6 = 54$点
\end{description}
\end{document}

\documentclass{ltjsarticle}
\usepackage[ipaex]{luatexja-preset}
\usepackage{amsmath}

\title{北京麻将}
\begin{document}\maketitle
\section{一般規則と取り決め}
北京 (bei3jing1) 麻将は「荘家門清没混提溜」といわれる。
親はつねに点数が2倍になる。
門前清なら点和でもアガれるが、ポン・チーをすると摸和でなければアガれない。
さらに、門前清でなければ、同じ手でも得点が半減してしまう。
没混とは混がないことであり、点数が2倍になる。
混は指標牌\footnote{開門についての記述がよくわからない}の次の牌\footnote{9は1へ・東南西北・中発白の順。}である。
提溜はツモアガリのことである。
やはり点数が2倍になる。

北京人が麻将をするときにはグループごとにルールが違うので確認しておかなければならない点\footnote{具体的な例は省略}が多くある。
\section{特殊規則「焼荘」}
\footnote{「焼き親」?}
親の第一打と同じ牌を子3人が切ったとき、親は子へ「焼き親点」を支払う。
さらに次巡も同様になったらまた支払うが、途切れた後は再現しても支払わない。
流局したらこの点は免除される。
\section{互包}
あるプレイヤーが他のプレイヤーに3回チー・ポン・カンさせると、この2人には「互包」関係が成立し、一方がアガった点数はすべて他方が支払う。
両者が同じ牌でアガったら、その点数の差にこれが適用される。
もし和牌者が複数人と互包になっていれば、両方から得点するので、総得点は人数のぶん増える。
\section{その他の規定}
\begin{enumerate}
    \item 和牌型は基本形と七対のみ。
    \item 缺一門でなければアガれない。
    \item 1つの打牌に複数のアガリを認める。
    \item 同巡フリテンはアガれない。
    \item 混が3枚あるときには点和できない。
    \item 7トン残して流局だが、2つ以上のカンがあれば8トンである。
    \item 流局または3連和の次局は点数が2倍。
    \item 搶槓は放銃なので門前清でなければできない。
    \item アガリまたは流局で連荘。
\end{enumerate}
\section{牌}
136枚。サイコロ2個。
\section{役と得点}
\begin{description}
    \item[門前清] 1点。
    \item[无混] 1点。
    \item[対々和] 2点。
    \item[全求人] 2点。
    \item[坎5和] 2点。
    \item[海底撈月 (摸和)] 2点。摸和でなければツモ切らなければならず、点和は2倍。
    \item[風一色] 6点 (乱)。
    \item[混一色] 2点。
    \item[清一色] 6点。
    \item[缺一門] 1点。
    \item[清龍] 3点。
    \item[平和] 1点。\footnote{参照先が組合龍だが……?}
    \item[杠上開花] 5点。
    \item[七対] 4点。四帰一1つで8点、2つで16点\footnote{3つで32点?}
    \item[四混] 8点。
\end{description}
\section{計算例}
\subsection{例}
親で缺一門・无混・清龍をツモったとき、得点は $(1+1+3)\times2\times2 = 20$オールである。
\subsection{例}
子で缺一門・門前清・坎5和をツモったとき、得点は $(1+1+2)\times2 = 8/16$ である。
\end{document}
